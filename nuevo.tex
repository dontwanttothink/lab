\documentclass[twocolumn]{article}
\usepackage{polyglossia}
\usepackage{amsmath}
\usepackage{amssymb}
\usepackage{booktabs}
\usepackage{float}
\usepackage{graphicx}
\usepackage{lmodern}
\usepackage{microtype}
\usepackage{pgfplots}
\usepackage{tikz}
\usepackage{xcolor}
\usepackage[pdfborder={0 0 0}]{hyperref}

\setdefaultlanguage{spanish}
\numberwithin{table}{section}
\definecolor{workblue}{HTML}{1377BA}
\pgfplotsset{compat=1.18}

\title{
  El Movimiento de un Péndulo: Una Comparación con el Movimiento
  Armónico Simple \\
  {\large Colegio San Jorge de Inglaterra \\
  Undécimo B}
}
\author{
  María Alejandra Andrade Sánchez \and
  Belcar Santiago Cuentas-Zavala Infante \and
  Juan Esteban Guzmán Garzón \and
  María Juliana Medina Higuera \and
  Jerónimo Rodríguez Garzón \and
  Juliana Rubiano Cabrera
}
\date{\today}

\begin{document}
\maketitle

\newpage
\tableofcontents

\begin{abstract}
  {\color{workblue} TODO\@: Adaptar al nuevo enfoque.}
\end{abstract}

\section{Introducción}

{\color{workblue}TODO\@: Adaptar al nuevo enfoque.}

\section{Marco teórico}

{\color{workblue}TODO\@: ADD THIS IN THE RIGHT PLACE.}
\begin{equation}
  A = L\theta_{\max}
\end{equation}

El movimiento armónico simple es una oscilación en la que la fuerza restauradora
es proporcional al desplazamiento. En el caso del péndulo, la fuerza
restauradora es dada por:

\begin{equation}\label{eq:restauradora}
  F_{r} = -mg\sin{\theta}
\end{equation}

Donde $m$ es la masa del objeto, $g$ es la aceleración gravitacional y $\theta$
es el ángulo de desplazamiento.

Aunque la ecuación~\eqref{eq:restauradora} no es lineal, es muy cercana
a una para pequeñas amplitudes.

\begin{equation*}
  \sin{\theta} \approx \theta \qquad\qquad \theta < 15^{\circ}
\end{equation*}

Esto permite obtener una fórmula lineal que aproxima la fuerza restauradora:

\begin{equation}\label{eq:restauradora_aprox_por_alpha}
  F_{r} \approx -mg\theta \qquad\qquad \theta < 15^{\circ}
\end{equation}

Por eso, se espera que el péndulo se comporte como un movimiento
armónico simple para pequeñas amplitudes, como las utilizadas aquí.

Comparando la ecuación~\eqref{eq:restauradora_aprox_por_alpha} con la
ecuación general de la fuerza restauradora en un movimiento armónico
simple, la fuerza restauradora es aproximadamente:

\begin{equation}\label{eq:restauradora_mas}
  F_{r} \approx -ks
\end{equation}

Donde:

\begin{itemize}
  \item $k = \frac{mg}{l}$ y $l$ es la longitud de la cuerda.
  \item $s = l\theta$, el desplazamiento a lo largo del
    arco.\ \theta{} se mide en radianes.
\end{itemize}

El experimento se puede analizar a partir de esta simplificación
porque se utilizaron únicamente ángulos menores a $15^{\circ}$, a
excepción de la medida de $18^{\circ}$, durante la variación de la amplitud.

\subsection{Periodo}

El periodo de un objeto en movimiento armónico simple puede ser
analizado a través de una comparación con el movimiento circular
uniforme. En particular, se observa la proyección de un movimiento
circular uniforme sobre el eje $x$.

\begin{figure}[ht]
  \centering
  \begin{tikzpicture}
    % Axis (dashed)
    \draw[dashed, thick] (-2.5,0) -- (2.5,0);

    % Circle
    \draw[] (0,0) circle (2cm);

    % Velocity
    \definecolor{velocitygreen}{rgb}{0,0.5,0}
    % tangent to the circle
    \draw[->, >=stealth, velocitygreen, line width=1mm] (60:2cm) --
    +({60+90}:2cm) node[above] {$v_{\max}$};
    % parallel to the x-axis
    \draw[->, >=stealth, velocitygreen, line width=1mm] (60:2cm) --
    +(0:-sin{60} * 2cm) coordinate (vend) node[below] {$v$};
    % dashed line perpendicular to the x-axis
    \draw[dashed] (vend) -- +(90:cos{60} * 2cm) coordinate (v_angle);
    % theta in velocity
    \draw (v_angle) ++(0,-3mm) arc (270:{270+60}:3mm) node[midway,
    below] {$\theta$};
    % fix transparency because i'm lazy
    \draw[->, >=stealth, velocitygreen, line width=1mm] (60:2cm) --
    +({60+90}:2cm) node[above] {$v_{\max}$};

    % Triangle
    % TODO (maybe): add a white rectangle so that the square root node doesn't
    % overlap the circle
    \draw (0,0) -- node[left] {$A$} (60:2cm) -- +(0, -2 * sin{60})
    node[right, midway] {$\sqrt{A^{2}-x^{2}}$};

    \fill[color={rgb:red,200;green,50;blue,0}] (60:2cm) circle (1.5pt);
    \draw (0:0.5cm) arc (0:60:0.5cm) node[midway, right] {$\theta$};

    % Horizontal displacement
    \draw[<->] (-0.5mm,-1mm) -- node[below] {$x$} +(0:2.17cm * cos{60});

    % Frame of reference
    \draw[->, >=stealth] (-2.5cm,-2.5cm) -- +(0,1cm) node[above] {$y$};
    \draw[->, >=stealth] (-2.5cm,-2.5cm) -- +(1cm,0) node[right] {$x$};
  \end{tikzpicture}
  \caption{Vista de un movimiento circular uniforme desde arriba}\label{fig:mcu}
\end{figure}

En la figura~\ref{fig:mcu}, tanto la velocidad como la posición de la
partícula en movimiento se describen con triángulos. Debido a que
estos triángulos comparten, invariablemente, tanto el ángulo theta
como un ángulo recto, ambos son similares. Este hecho se expresa en
la ecuación~\eqref{eq:triángulos}.

\begin{align}
  \frac{v}{v_{\max}} = \frac{\sqrt{A^{2} - x^{2}}}{A} \nonumber &\\
  v = v_{\max}\sqrt{1 - \frac{x^{2}}{A^{2}}} & \label{eq:triángulos}
\end{align}

Esta misma ecuación puede obtenerse a través de un análisis
energético del movimiento armónico simple:

\begin{align}
  \frac{1}{2}mv^{2} + \frac{1}{2}kx^{2} &= \frac{1}{2}kA^{2} \nonumber \\
  v^{2} &= \frac{k}{m}(A^{2} - x^{2}) \nonumber \\
  v^{2} &= \frac{k}{m}A^{2}\left(1 - \frac{x^{2}}{A^{2}}\right)
  \label{eq:velocidad_amplitud}
\end{align}

La velocidad máxima se obtiene cuando $x = 0$; es decir, en el punto de
equilibrio. Por lo tanto,

\begin{align}
  \frac{1}{2}mv_{\max}^{2} + \frac{1}{2}k(0^{2}) &= \frac{1}{2}kA^{2}
  \nonumber \\
  v_{\max}^{2} &= \frac{k}{m}A^{2} \label{eq:velocidad_max}
\end{align}

Se pueden combinar las ecuaciones~\eqref{eq:velocidad_amplitud}
y~\eqref{eq:velocidad_max}, dando como resultado:

\begin{align}
  v^{2} = v^{2}_{\max}\left(1 - \frac{x^{2}}{A^{2}}\right) \nonumber \\
  v = \pm v_{\max}\sqrt{1 - \frac{x^{2}}{A^{2}}}
\end{align}

Debido a que la velocidad es igual en cada punto, la proyección en el
eje $x$ de un movimiento circular uniforme es idéntica a un movimiento armónico
simple. Aquello permite obtener el periodo de un movimiento armónico
simple, porque es igual al tiempo que tardaría en dar un ciclo una
partícula en el movimiento circular uniforme correspondiente.

\begin{align}
  T &= \frac{2\pi}{v_{\max}} \nonumber \\
  T &= 2\pi\sqrt{\frac{m}{k}} \quad \text{Ya que $v_{\max} =
  \sqrt{\frac{k}{m}}A$} \nonumber \\
  T &= 2\pi\sqrt{\frac{l}{g}} \quad \text{Ya que $k =
  \frac{mg}{l}$}\label{eq:periodo}
\end{align}

\subsection{Frecuencia}

En el péndulo simple, la frecuencia se refiere al número de veces que el péndulo
completa un ciclo en un segundo. Es igual a $\frac{1}{T}$.

Combinando la fórmula del periodo según la aproximación basada en
el movimiento armónico simple~\eqref{eq:periodo} en la ecuación de la
frecuencia, se obtiene:

\begin{equation}
  f = \frac{1}{2\pi}\sqrt{\frac{g}{l}}
\end{equation}

\subsection{Velocidad}

\subsubsection{Velocidad angular}

La velocidad angular es el ángulo recorrido por un objeto por unidad
de tiempo, y se mide en radianes por segundo.

\begin{equation}
  \omega = \frac{\theta}{t} \label{eq:velocidad_angular}
\end{equation}

También,

\begin{equation}
  \omega = \frac{2\pi}{T}
\end{equation}

Donde $T$ es el periodo.

\subsubsection{Velocidad máxima}

Como se mencionó anteriormente, la velocidad máxima según la
aproximación con el movimiento armónico simple está dada por:

\begin{equation}
  v_{\max} = \sqrt{\frac{k}{m}}A \tag{\ref{eq:velocidad_max}}
\end{equation}

La velocidad varía a lo largo de la oscilación. Es máxima en el punto
de equilibrio, donde toda la energía potencial se convierte en
energía cinética, y es cero en los extremos, donde la energía
cinética se transforma en energía potencial. Dado que la velocidad es
un vector, se asigna un signo positivo al movimiento hacia un lado
(en este caso, hacia la derecha) y negativo hacia el lado opuesto
(hacia la izquierda), considerando el punto de equilibrio como el origen.

\subsection{Energía}

\subsubsection{Energía potencial}

La energía potencial gravitacional del péndulo es máxima en los
extremos de su movimiento, cuando alcanza su punto más alto, y mínima
cuando se encuentra en el punto de equilibrio. Se define por la
ecuación~\eqref{eq:potencial_gravitacional}.

\begin{equation}
  E_{p} = mgh \label{eq:potencial_gravitacional}
\end{equation}

\subsubsection{Energía cinética}

La energía cinética del péndulo es máxima en el punto de equilibrio,
cuando el péndulo pasa por la vertical, y mínima en los extremos de
su movimiento. Se define por la ecuación~\eqref{eq:cinética}.

\begin{equation}
  E_{c} = \frac{1}{2}mv^{2} \label{eq:cinética}
\end{equation}

\subsubsection{Energía mecánica}

La energía mecánica es la suma de la energía cinética y la energía
potencial de un sistema. En el caso del péndulo, la energía mecánica
se mantiene constante durante el movimiento, ya que la fricción es
negligible y no hay pérdidas significativas de energía debido a
fuerzas externas y no conservativas.

\begin{align}
  E_{m} &= E_{c} + E_{p} \nonumber \\
  E_{m} &= \frac{1}{2}mv^{2}
\end{align}

\subsection{Funciones del tiempo}

\subsubsection{Desplazamiento en función del tiempo}

En la figura~\ref{fig:mcu}, se observa que:

\begin{equation}
  x = A\cos{\theta} \label{eq:desplazamiento_ángulo}
\end{equation}

Reescribiendo la ecuación~\eqref{eq:velocidad_angular}, obtenemos que:

\begin{equation*}
  \theta = \omega t
\end{equation*}

Sustituyendo este valor en la
ecuación~\eqref{eq:desplazamiento_ángulo}, se obtiene:

\begin{equation}
  s(t) = A\cos{(\omega t)} \label{eq:desplazamiento_tiempo}
\end{equation}

Aquí, $s(t)$ representa el desplazamiento del objeto en un movimiento
armónico simple después de un tiempo $t$ y donde $A$ es la amplitud;
es decir, el máximo desplazamiento del objeto.

Para obtener el ángulo recorrido por el objeto en un tiempo $t$, se
puede escribir una nueva ecuación:

\begin{equation}
  \theta(t) = \theta_{\max}\cos(\omega t) \label{eq:ángulo_tiempo}
\end{equation}

\subsubsection{Velocidad en función del tiempo}

Dado que la variación de $s$ con respecto al tiempo $t$ es la
velocidad, la derivada de la ecuación de desplazamiento con respecto
al tiempo~\eqref{eq:desplazamiento_tiempo} da la velocidad en función
del tiempo. Esto se expresa como:

\begin{align}
  v(t) &= \frac{d}{dt} \left[s(t)\right] \nonumber \\
  v(t) &= \frac{d}{dt} \left[A\cos{(\omega t)}\right] \nonumber \\
  v(t) &= -A\omega\sin{(\omega t)}\label{eq:velocidad_tiempo}
\end{align}

\subsubsection{Aceleración en función del tiempo}

Dado que la variación de la velocidad $v$ con respecto al tiempo $t$
es la aceleración, la derivada de la velocidad con respecto al tiempo
da la aceleración en función del tiempo. Esto se expresa como:

\begin{align}
  a(t) &= \frac{d^{2}}{dt^{2}} \left[s(t)\right] \nonumber \\
  a(t) &= \frac{d}{dt} \left[v(t)\right] \nonumber \\
  a(t) &= \frac{d}{dt} \left[-A\omega\sin{(\omega t)}\right] \nonumber \\
  a(t) &= -A\omega^{2}\cos{(\omega t)}\label{eq:aceleración_tiempo}
\end{align}

\section{Formulación de hipótesis}

Se plantean las siguientes hipótesis para este experimento:

\begin{itemize}
  \item Se espera que exista una correlación positiva entre la longitud
    de la cuerda y el periodo.
  \item Se espera que la variación de la amplitud del péndulo no tenga un
    efecto significativo sobre el periodo.
  \item Se espera que el periodo observado coincida con el esperado para el
    movimiento armónico simple.
  \item Se espera que la variación de la masa sea irrelevante al periodo
    de oscilación del péndulo.
\end{itemize}

\section{Diseño experimental}

Una varilla se fijó perpendicularmente a un soporte. Se ató un cuerda a
la varilla, que a la vez se ató a una masa. El aparato se ilustra en la
figura~\ref{fig:diseño}.

\begin{figure}[ht]
  \centering
  \begin{tikzpicture}
    % Mesa
    \draw (0,0) rectangle (4cm, 3mm);
    \draw (5mm, 0) rectangle ++(1mm, -8mm);
    \draw ({4cm - 6mm}, 0) rectangle ++(1mm, -8mm);

    % Soporte universal
    \draw (2.5cm, 3mm) rectangle ++(1.2cm, 1mm) coordinate (S);
    \draw (S) ++({-0.6cm - 0.75mm}, 0) rectangle ++(1.5mm, 2.5cm)
    coordinate (T)
    node[midway, left, xshift=-1mm]
    {soporte};
    \draw (T) ++(-3mm, -1mm) coordinate (U) rectangle ++(1cm, 0.5mm);
    \fill[white] (S) ++({-0.6cm - 0.75mm}, 0) rectangle ++(1.5mm, 2.5cm);

    % cuerda
    \draw (U) ++(0.7cm, 0) -- ++({270+15}:2cm) coordinate (V)
    node[midway, right]
    {cuerda};

    % Masa
    \draw (V) ++(-1.25mm, 0) rectangle ++(2.5mm, -0.8mm) coordinate (W);
    \draw (W) ++(0.5mm, 0) -- ++(-0.5mm - 2.5mm - 0.5mm, 0)
    node[midway, below]
    {masa};
  \end{tikzpicture}
  \caption{Diseño experimental}\label{fig:diseño}
\end{figure}

\subsection{Materiales necesarios}

Se necesitan:

\begin{itemize}
  \item cuerda;
  \item un cronómetro;
  \item un soporte para el péndulo;
  \item pesas de $0.05$\,kg, $0.10$\,kg, $0.20$\,kg, y $0.25$\,kg;
  \item una regla;
  \item un transportador;
  \item una balanza;
  \item y varillas.
\end{itemize}

\subsection{Variables y constantes}

{\color{workblue}TODO\@: Robby quiere tablas para cada uno de los
tres experimentos.}

En el experimento, se mantienen constantes:

\begin{itemize}
  \item la aceleración gravitacional y
  \item el material de la cuerda.
\end{itemize}

Para la prueba con masa variable, se mantienen constantes la longitud
de la cuerda y la amplitud. Para la prueba con amplitud variable, se
mantienen constantes la masa y la longitud de la cuerda. Y para la
prueba con longitud de cuerda variable, se mantienen constantes la
masa y la amplitud.

Por otro lado, se miden y controlan las variables como se indica en el
cuadro~\ref{fig:variables}.

\begin{table}[ht]
  \centering
  \begin{tabular}{lccc}
    \toprule
    \textbf{Variable} & \textbf{Dep.} & \textbf{Med.} &\textbf{Natur.} \\
    \midrule
    \textbf{Masa}                 & $I$& $D$&\triangle\\
    \textbf{Longitud de cuerda}   & $I$& $D$&\triangle\\
    \textbf{Ángulo (amplitud)}    & $I$& $D$&\triangle\\
    \textbf{Tiempo}               & $D$& $D$&\triangle\\
    \bottomrule
  \end{tabular}
  \caption{Caracterización de las variables}\label{fig:variables}
  \vspace{0.5em}
  \begin{minipage}{\columnwidth}
    \footnotesize
    Las columnas abreviadas se refieren a (1) la dependencia o
    independencia de las variables, (2) si se miden directamente o
    indirectamente, y (3) si su naturaleza es cuantitativa o cualitativa.
    \vspace{0.5em}

    \textbf{Leyenda:} \\
    \begin{tabular}{cl}
      $\triangle$ & cuantitativa \\
      $\square$   & cualitativa \emph{(En desuso.)}
    \end{tabular}
  \end{minipage}
\end{table}

\section{Procedimiento}

En todos los casos, se calcularon promedios de los datos que se
midieron más de una vez. El periodo de oscilación se aproximó
dividiendo entre diez el tiempo promedio transcurrido en cada configuración
experimental.

\subsection{Variación de la longitud del cuerda}

\begin{enumerate}
  \item Se midió 1\,m de cuerda con una regla.\label{it:proc1_inicio}
  \item Se ató una masa de $0.05$\,kg a la cuerda.
  \item Se liberó la masa a una amplitud de $15$ grados, medida con
    el transportador, y simultáneamente se inició el
    cronómetro.\label{it:proc1_liberación}
  \item Después de diez oscilaciones, se pausó el cronómetro y se
    registró el tiempo transcurrido.\label{it:proc1_registro}
  \item Se repitieron los pasos~\ref{it:proc1_liberación}
    y~\ref{it:proc1_registro} tres veces.\label{it:proc1_fin}
  \item Se repitieron los pasos~\ref{it:proc1_inicio}
    a~\ref{it:proc1_fin}, utilizando longitudes de cuerda de
    $50$\,cm, $30$\,cm y $15$\,cm.
\end{enumerate}

\subsection{Variación de la amplitud}

\begin{enumerate}
  \item Se midieron 60\,cm de cuerda con una regla.\label{it:proc2_inicio}
  \item Se ató una masa de $0.10$\,kg a la cuerda.
  \item Se liberó la masa a una amplitud de $6$ grados, medida con
    el transportador, y simultáneamente se inició el
    cronómetro.
  \item Después de diez oscilaciones, se pausó el cronómetro y se
    registró el tiempo transcurrido.\label{it:proc2_registro}
  \item Se repitieron los pasos~\ref{it:proc1_liberación}
    y~\ref{it:proc2_registro} tres veces.\label{it:proc2_fin}
  \item Se repitieron los pasos~\ref{it:proc2_inicio}
    a~\ref{it:proc2_fin}, utilizando amplitudes de
    $9$, $12$, $15$ y $18$ grados.
\end{enumerate}

\subsection{Variación de la masa}

\begin{enumerate}
  \item Se midieron 60\,cm de cuerda con una regla.\label{it:proc3_inicio}
  \item Se ató una masa de $0.05$\,kg a la cuerda.
  \item Se liberó la masa a una amplitud de $10$ grados, medida con
    el transportador, y simultáneamente se inició el
    cronómetro.
  \item Después de diez oscilaciones, se pausó el cronómetro y se
    registró el tiempo transcurrido.\label{it:proc3_registro}
  \item Se repitieron los pasos~\ref{it:proc1_liberación}
    y~\ref{it:proc3_registro} tres veces.\label{it:proc3_fin}
  \item Se repitieron los pasos~\ref{it:proc3_inicio}
    a~\ref{it:proc3_fin}, utilizando masas de
    $0.10$\,kg, $0.15$\,kg, $0.20$\,kg y $0.25$\,kg.
\end{enumerate}

\newpage
\onecolumn
\section{Resultados}

\begin{table}[ht]
  \centering
  \caption{Efecto de la longitud en el periodo}\label{tab:longitud_periodo}
  \begin{tabular}{cccccc}
    \toprule
    Longitud $L$ (m) & \multicolumn{3}{c}{Tiempo para 10
    oscilaciones (s)} & Tiempo promedio (s) & Periodo $T$ (s) \\
    \cmidrule(lr){2-4}
    & Medición 1 & Medición 2 & Medición 3 &  &  \\
    \midrule
    0.15 &  9.56 &  9.68 &  9.71 &  9.65 & 0.965 \\
    0.30 & 12.39 & 12.69 & 12.98 & 12.68 & 1.269 \\
    0.50 & 15.32 & 15.65 & 15.66 & 15.54 & 1.554 \\
    1.00 & 20.78 & 20.76 & 20.91 & 20.82 & 2.082 \\
    \bottomrule
  \end{tabular}
\end{table}

\begin{table}[ht]
  \centering
  \caption{Efecto de la amplitud en el periodo}\label{tab:amplitud_periodo}
  \begin{tabular}{cccccc}
    \toprule
    Amplitud $\theta$ (°) & \multicolumn{3}{c}{Tiempo para 10
    oscilaciones (s)} & Tiempo promedio (s) & Periodo $T$ (s) \\
    \cmidrule(lr){2-4}
    & Medición 1 & Medición 2 & Medición 3 &  &  \\
    \midrule
    6   & 16.39 & 16.56 & 16.76 & 16.57 & 1.657 \\
    9   & 16.41 & 16.95 & 16.69 & 16.68 & 1.668 \\
    12  & 17.01 & 16.80 & 17.09 & 16.97 & 1.697 \\
    15  & 16.57 & 16.72 & 17.00 & 16.76 & 1.676 \\
    18  & 16.41 & 16.53 & 16.72 & 16.55 & 1.655 \\
    \bottomrule
  \end{tabular}
\end{table}

\begin{table}[ht]
  \centering
  \caption{Efecto de la masa en el periodo}\label{tab:masa_periodo}
  \begin{tabular}{cccccc}
    \toprule
    Masa (kg) & \multicolumn{3}{c}{Tiempo para 10 oscilaciones
    (s)} & Tiempo promedio (s) & Periodo $T$ (s) \\
    \cmidrule(lr){2-4}
    & Medición 1 & Medición 2 & Medición 3 &  &  \\
    \midrule
    0.05  & 16.52 & 16.5  & 16.78 & 16.60 & 1.660 \\
    0.10  & 16.53 & 16.68 & 17.09 & 16.76 & 1.677 \\
    0.15  & 16.68 & 16.70 & 17.00 & 16.79 & 1.679 \\
    0.20  & 16.68 & 16.71 & 16.95 & 16.78 & 1.678 \\
    0.25  & 16.91 & 16.77 & 16.74 & 16.81 & 1.681 \\
    \bottomrule
  \end{tabular}
\end{table}

\twocolumn

\subsection{Limitaciones}

% todo message in blue
\textcolor{workblue}{Se debe explicar el significado de los porcentajes de
error.}

Durante el experimento, se tomaron múltiples mediciones bajo las mismas
condiciones. Si el experimento hubiera estado libre de errores, estas
mediciones serían iguales. Sin embargo, los errores humanos e inexactitudes
inevitables resultan en una ligera variación. Algunas fuentes de
error posibles incluyen:

\begin{itemize}
  \item el efecto de rozamientos mínimos e imperceptibles sobre las
    oscilaciones del péndulo,
  \item longitudes mal medidas o inexactas,
  \item mediciones de tiempo inexactas\footnote{Algunos cronómetros
    empiezan a contar unas décimas de segundo después o antes.}
  \item y mediciones de ángulo inexactas o ángulos mal medidos.
\end{itemize}

Se elaboraron los cuadros~\ref{tab:error_longitud}~a~\ref{tab:error_amplitud}
para mostrar esta variación, que establece la existencia de errores e
inexactitudes.

El porcentaje de error está dado por la fórmula:

\begin{equation}
  \text{Error} = \frac{T_{\text{prom}} -
  T_{\text{medido}}}{T_{\text{prom}}} \times 100
\end{equation}

Donde $T_{\text{prom}}$ es el periodo promedio y $T_{\text{medido}}$ es
el periodo medido.

\begin{table}[ht]
  \centering
  \begin{tabular}{cccc}
    \toprule
    Longitud $L$ (m) & \multicolumn{3}{c}{\% de error para $10$
    oscilaciones} \\
    \cmidrule(lr){2-4}
    & Med. 1 & Med. 2 & Med. 3  \\
    \midrule
    0.15 & 0.93 & 0.31 & 0.62 \\
    0.30 & 2.63 & 0.06 & 2.28 \\
    0.50 & 1.42 & 0.71 & 0.77 \\
    1.00 & 0.19 & 0.29 & 0.43 \\
    \bottomrule
  \end{tabular}
  \caption{Errores en la medición durante la variación de la
  longitud del cuerda}\label{tab:error_longitud}
\end{table}

\begin{table}[ht]
  \centering
  \begin{tabular}{cccc}
    \toprule
    Masa $m$ (kg) & \multicolumn{3}{c}{\% de error para $10$
    oscilaciones} \\
    \cmidrule(lr){2-4}
    & Med. 1 & Med. 2 & Med. 3  \\
    \midrule
    0.05 & 0.48 & 0.60 & 1.08 \\
    0.10 & 1.43 & 0.54 & 1.91 \\
    0.15 & 0.68 & 0.56 & 1.23 \\
    0.20 & 0.60 & 0.42 & 1.01 \\
    0.25 & 0.59 & 0.24 & 0.42 \\
    \bottomrule
  \end{tabular}
  \caption{Errores en la medición durante la variación de la
  masa de la pesa}\label{tab:error_masa}
\end{table}

\begin{table}[ht]
  \centering
  \begin{tabular}{cccc}
    \toprule
    Amplitud $\theta$ ($^{\circ}$) & \multicolumn{3}{c}{\% de
      error para $10$
    oscilaciones} \\
    \cmidrule(lr){2-4}
    & Med. 1 & Med. 2 & Med. 3  \\
    \midrule
    6  & 1.09 & 0.06 & 1.15 \\
    9  & 1.64 & 1.60 & 0.04 \\
    12 & 1.79 & 0.31 & 1.73 \\
    15 & 1.13 & 0.24 & 1.43 \\
    18 & 0.85 & 0.12 & 1.03 \\
    \bottomrule
  \end{tabular}
  \caption{Errores en la medición durante la variación de la
  amplitud de liberación}\label{tab:error_amplitud}
\end{table}

\section{Interpretación de los resultados}

\subsection{Variación de la longitud}

\begin{table}
  \centering
  \begin{tabular}{cccc}
    \toprule
    $L$ (m) & $T_{\text{m}}$ (s) &
    $T_{\text{e}}$ (s) & \% de variación \\
    \midrule
    0.15 & 0.965 & 0.965 & 0.00 \\
    0.30 & 1.269 & 1.269 & 0.00 \\
    0.50 & 1.554 & 1.554 & 0.00 \\
    1.00 & 2.082 & 2.082 & 0.00 \\
    \bottomrule
  \end{tabular}

  \caption{Comparación entre los valores medidos y esperados para $T$
    según el movimiento armónico simple, con porcentajes de
  variación}\label{tab:comparación_longitud}

  \vspace{0.5em}
  \begin{minipage}{\columnwidth}
    \footnotesize
    $T_{\text{m}}$ se refiere al periodo medido y $T_{\text{e}}$ al
    periodo esperado.
  \end{minipage}
\end{table}

En la tabla~\ref{tab:comparación_longitud}, se comparan los valores medidos
y esperados para el periodo $T$ en función de la longitud de la
cuerda. El porcentaje de variación se calcula con la fórmula:

\begin{equation}
  \text{Variación} = \frac{T_{\text{m}} - T_{\text{e}}}{T_{\text{e}}} \times 100
\end{equation}

Donde $T_{e}$ es el periodo esperado y $T_{m}$ es el periodo medido.

\begin{figure}[ht]
  \centering
  \begin{tikzpicture}
    \begin{axis}[
        xlabel={Longitud $L$ (m)},
        ylabel={Periodo $T$ (s)},
        grid=both,
        xscale=0.85,
        legend style={at={({1/0.85},1.25)}},
        xmin=0, xmax=1,
        ymin=0, ymax=2.5,
        domain=0:1,
        samples=200,
      ]

      % Expected values
      \addplot[red] {2*pi*sqrt(x/9.81)};
      \addlegendentry{Periodo esperado};

      % Data points
      \addplot[only marks, mark=o]
      coordinates {(0.15, 0.965) (0.30, 1.269) (0.50, 1.554) (1.00, 2.082)};
      \addlegendentry{Periodo medido};
    \end{axis}
  \end{tikzpicture}
  \caption{Periodo $T$ medido y esperado para un movimiento armónico
    simple de las mismas características
    $\left(2\pi\sqrt{\frac{L}{g}}\right)$ en función de la longitud de
  la cuerda $L$}\label{fig:periodo_mas}
\end{figure}

\subsection{Variación de la masa}

\subsection{Variación de la amplitud}

% TODO: Finish this section

\subsection{Conclusiones}

{\color{workblue} TODO\@: Adaptar al nuevo enfoque.}

\renewcommand{\refname}{Bibliografía}
\begin{thebibliography}{1}
  \bibitem{}
  Giancoli, D.\,C. (2014) \textit{Physics: Principles and
  Applications}. Pearson Education, Inc.
\end{thebibliography}

\end{document}
