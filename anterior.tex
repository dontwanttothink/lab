\documentclass[twocolumn]{article}
\usepackage{amsmath}
\usepackage{amssymb}
\usepackage{booktabs}
\usepackage{float}
\usepackage{graphicx}
\usepackage[pdfborder={0 0 0}]{hyperref}
\usepackage{microtype}
\usepackage{pgfplots}
\usepackage{polyglossia}
\usepackage{tikz}
\usepackage{xcolor}

\setdefaultlanguage{spanish}
\numberwithin{table}{section}
\definecolor{workblue}{HTML}{1377BA}
\pgfplotsset{compat=1.18}

\title{
  El Movimiento de un Péndulo: Un Estudio Teórico y Experimental,
  con Comparación con el Movimiento Armónico Simple \\
  {\large Colegio San Jorge de Inglaterra \\
  Undécimo B}
}
\author{
  María Alejandra Andrade Sánchez \and
  Belcar Santiago Cuentas-Zavala Infante \and
  Juan Esteban Guzmán Garzón \and
  María Juliana Medina Higuera \and
  Jerónimo Rodríguez Garzón \and
  Juliana Rubiano Cabrera
}
\date{\today}

\begin{document}
\maketitle

\newpage
\tableofcontents

\begin{abstract}
  Se estudió el movimiento de un péndulo: un cuerpo suspendido de
  una cuerda unida a un soporte fijo y que se balancea de un lado a
  otro por el efecto de la gravedad. Primero, se analizó el
  sistema desde la mecánica clásica y la energía. Luego, se midió su
  periodo de oscilación mientras se variaba la masa del cuerpo, la
  amplitud desde la que se liberó y la longitud de la cuerda que lo
  sostenía. La evidencia experimental junta con el análisis
  teórico permitió concluir que el periodo de oscilación es
  independiente de la masa del cuerpo suspendido y que el periodo
  está correlacionado positivamente con la longitud de la cuerda.
  El análisis teórico mostró que existe una correlación positiva
  entre la amplitud de liberación del cuerpo suspendido y su
  velocidad máxima. No se pudo establecer una relación directa
  entre la amplitud y el periodo. Finalmente, se consideraron los
  errores experimentales posibles, como la imprecisión en el uso de
  los cronómetros, la lectura de los ángulos con el transportador y
  la medición de la longitud de la cuerda. Se presentó el
  porcentaje de deviación relativa de cada medición.
\end{abstract}

\section{Introducción}

En este informe de laboratorio, se buscó analizar el movimiento
oscilatorio de un péndulo y las variables que influyen en su
comportamiento. Esto se llevó a cabo midiendo el tiempo que tarda un
péndulo en realizar diez oscilaciones, variando factores como la
masa, la longitud de la cuerda y la amplitud del movimiento. Además,
se pretende establecer relaciones matemáticas entre estas variables
(masa, tiempo y longitud de la cuerda) y el periodo de oscilación del
péndulo.

Algunas preguntas que guiaron la investigación son: ¿Cómo
afecta la longitud de la cuerda al periodo de oscilación del péndulo?
¿Qué impacto tiene la masa colgante en el movimiento oscilatorio? ¿De
qué manera influye la amplitud en la velocidad máxima del péndulo
durante una oscilación?

\section{Formulación de la hipótesis}
Se tienen las siguientes expectativas sobre el movimiento del péndulo,
en parte basado en el análisis teórico.\footnote{Vea la
sección~\ref{sec:álgebra}.}

\subsection*{El efecto de la longitud de la cuerda sobre el
periodo de oscilación}
Se hipotetiza que existe una correlación positiva entre la longitud
de la cuerda y la duración de las oscilaciones.

Esta evaluación se basa en la proporcionalidad directa entre la
longitud de la cuerda y la longitud del arco a través del cual ocurre
el movimiento.

\subsection*{El efecto de la variación de la masa colgante sobre
el periodo de movimiento del péndulo}

Se hipotetiza que la variación de la masa es irrelevante al periodo
de oscilación del péndulo.

\subsection*{El efecto de la variación de la amplitud de liberación
sobre el movimiento del péndulo}

Se hipotetiza que existe una correlación positiva entre la amplitud y
la velocidad máxima del péndulo durante una oscilación.

\subsection*{La relación con el movimiento armónico simple}
Se hipotetiza que el movimiento del péndulo puede aproximarse con una
buena precisión mediante el modelo del movimiento armónico simple.

\section{Diseño experimental}

Una varilla se fijó perpendicularmente a un soporte. Se ató un hilo a
la varilla, que a la vez se ató a una masa. El aparato se ilustra en la
figura~\ref{fig:diseño}.

\begin{figure}[ht]
  \centering
  \begin{tikzpicture}
    % Mesa
    \draw (0,0) rectangle (4cm, 3mm);
    \draw (5mm, 0) rectangle ++(1mm, -8mm);
    \draw ({4cm - 6mm}, 0) rectangle ++(1mm, -8mm);

    % Soporte universal
    \draw (2.5cm, 3mm) rectangle ++(1.2cm, 1mm) coordinate (S);
    \draw (S) ++({-0.6cm - 0.75mm}, 0) rectangle ++(1.5mm, 2.5cm)
    coordinate (T)
    node[midway, left, xshift=-1mm]
    {soporte};
    \draw (T) ++(-3mm, -1mm) coordinate (U) rectangle ++(1cm, 0.5mm);
    \fill[white] (S) ++({-0.6cm - 0.75mm}, 0) rectangle ++(1.5mm, 2.5cm);

    % Hilo
    \draw (U) ++(0.7cm, 0) -- ++({270+15}:2cm) coordinate (V)
    node[midway, right]
    {hilo};

    % Masa
    \draw (V) ++(-1.25mm, 0) rectangle ++(2.5mm, -0.8mm) coordinate (W);
    \draw (W) ++(0.5mm, 0) -- ++(-0.5mm - 2.5mm - 0.5mm, 0)
    node[midway, below]
    {masa};
  \end{tikzpicture}
  \caption{Diseño experimental}\label{fig:diseño}
\end{figure}

\subsection{Materiales necesarios}

Se necesitan:

\begin{itemize}
  \item hilo;
  \item un cronómetro;
  \item un soporte para el péndulo;
  \item pesas de $0.05$\,kg, $0.10$\,kg, $0.20$\,kg, y $0.25$\,kg;
  \item una regla;
  \item un transportador;
  \item una balanza;
  \item y varillas.
\end{itemize}

\subsection{Variables y constantes}

En el experimento, se mantienen constantes:

\begin{itemize}
  \item la aceleración gravitacional y
  \item el material de la cuerda.
\end{itemize}

Para la prueba con masa variable, se mantienen constantes la longitud
de la cuerda y la amplitud. Para la prueba con amplitud variable, se
mantienen constantes la masa y la longitud de la cuerda. Y para la
prueba con longitud de cuerda variable, se mantienen constantes la
masa y la amplitud.

Por otro lado, se miden y controlan las variables como se indica en el
cuadro~\ref{fig:variables}.

\begin{table}[ht]
  \centering
  \begin{tabular}{lccc}
    \toprule
    \textbf{Variable} & \textbf{Dep.} & \textbf{Med.} &\textbf{Natur.} \\
    \midrule
    \textbf{Masa}                 & $I$& $D$&\triangle\\
    \textbf{Longitud de cuerda}   & $I$& $D$&\triangle\\
    \textbf{Ángulo (amplitud)}    & $I$& $D$&\triangle\\
    \textbf{Tiempo}               & $D$& $D$&\triangle\\
    \bottomrule
  \end{tabular}
  \caption{Caracterización de las variables}\label{fig:variables}
  \vspace{0.5em}
  \begin{minipage}{\columnwidth}
    \footnotesize
    Las columnas abreviadas se refieren a (1) la dependencia o
    independencia de las variables, (2) si se miden directamente o
    indirectamente, y (3) si su naturaleza es cuantitativa o cualitativa.
    \vspace{0.5em}

    \textbf{Leyenda:} \\
    \begin{tabular}{cl}
      $\triangle$ & cuantitativa \\
      $\square$   & cualitativa \emph{(En desuso.)}
    \end{tabular}
  \end{minipage}
\end{table}

\section{Procedimiento}

En todos los casos, se calcularon promedios de los datos que se
midieron más de una vez. El periodo de oscilación se aproximó
dividiendo entre diez el tiempo promedio transcurrido en cada configuración
experimental.

\subsection{Variación de la longitud del hilo}

\begin{enumerate}
  \item Se midió 1\,m de hilo con una regla.\label{it:proc1_inicio}
  \item Se ató una masa de $0.05$\,kg al hilo.
  \item Se liberó la masa a una amplitud de $15$ grados, medida con
    el transportador, y simultáneamente se inició el
    cronómetro.\label{it:proc1_liberación}
  \item Después de diez oscilaciones, se pausó el cronómetro y se
    registró el tiempo transcurrido.\label{it:proc1_registro}
  \item Se repitieron los pasos~\ref{it:proc1_liberación}
    y~\ref{it:proc1_registro} tres veces.\label{it:proc1_fin}
  \item Se repitieron los pasos~\ref{it:proc1_inicio}
    a~\ref{it:proc1_fin}, utilizando longitudes de hilo de
    $50$\,cm, $30$\,cm y $15$\,cm.
\end{enumerate}

\subsection{Variación de la amplitud}

\begin{enumerate}
  \item Se midieron 60\,cm de hilo con una regla.\label{it:proc2_inicio}
  \item Se ató una masa de $0.10$\,kg al hilo.
  \item Se liberó la masa a una amplitud de $6$ grados, medida con
    el transportador, y simultáneamente se inició el
    cronómetro.
  \item Después de diez oscilaciones, se pausó el cronómetro y se
    registró el tiempo transcurrido.\label{it:proc2_registro}
  \item Se repitieron los pasos~\ref{it:proc1_liberación}
    y~\ref{it:proc2_registro} tres veces.\label{it:proc2_fin}
  \item Se repitieron los pasos~\ref{it:proc2_inicio}
    a~\ref{it:proc2_fin}, utilizando amplitudes de
    $9$, $12$, $15$ y $18$ grados.
\end{enumerate}

\subsection{Variación de la masa}

\begin{enumerate}
  \item Se midieron 60\,cm de hilo con una regla.\label{it:proc3_inicio}
  \item Se ató una masa de $0.05$\,kg al hilo.
  \item Se liberó la masa a una amplitud de $10$ grados, medida con
    el transportador, y simultáneamente se inició el
    cronómetro.
  \item Después de diez oscilaciones, se pausó el cronómetro y se
    registró el tiempo transcurrido.\label{it:proc3_registro}
  \item Se repitieron los pasos~\ref{it:proc1_liberación}
    y~\ref{it:proc3_registro} tres veces.\label{it:proc3_fin}
  \item Se repitieron los pasos~\ref{it:proc3_inicio}
    a~\ref{it:proc3_fin}, utilizando masas de
    $0.10$\,kg, $0.15$\,kg, $0.20$\,kg y $0.25$\,kg.
\end{enumerate}

\newpage
\onecolumn
\section{Resultados}

\begin{table}[ht]
  \centering
  \caption{Efecto de la longitud en el periodo}\label{tab:longitud_periodo}
  \begin{tabular}{cccccc}
    \toprule
    Longitud $L$ (m) & \multicolumn{3}{c}{Tiempo para 10
    oscilaciones (s)} & Tiempo promedio (s) & Periodo $T$ (s) \\
    \cmidrule(lr){2-4}
    & Medición 1 & Medición 2 & Medición 3 &  &  \\
    \midrule
    0.15 &  9.56 &  9.68 &  9.71 &  9.65 & 0.965 \\
    0.30 & 12.39 & 12.69 & 12.98 & 12.68 & 1.269 \\
    0.50 & 15.32 & 15.65 & 15.66 & 15.54 & 1.554 \\
    1.00 & 20.78 & 20.76 & 20.91 & 20.82 & 2.082 \\
    \bottomrule
  \end{tabular}
\end{table}

\begin{table}[ht]
  \centering
  \caption{Efecto de la amplitud en el periodo}\label{tab:amplitud_periodo}
  \begin{tabular}{cccccc}
    \toprule
    Amplitud $\alpha$ (°) & \multicolumn{3}{c}{Tiempo para 10
    oscilaciones (s)} & Tiempo promedio (s) & Periodo $T$ (s) \\
    \cmidrule(lr){2-4}
    & Medición 1 & Medición 2 & Medición 3 &  &  \\
    \midrule
    6   & 16.39 & 16.56 & 16.76 & 16.57 & 1.657 \\
    9   & 16.41 & 16.95 & 16.69 & 16.68 & 1.668 \\
    12  & 17.01 & 16.80 & 17.09 & 16.97 & 1.697 \\
    15  & 16.57 & 16.72 & 17.00 & 16.76 & 1.676 \\
    18  & 16.41 & 16.53 & 16.72 & 16.55 & 1.655 \\
    \bottomrule
  \end{tabular}
\end{table}

\begin{table}[ht]
  \centering
  \caption{Efecto de la masa en el periodo}\label{tab:masa_periodo}
  \begin{tabular}{cccccc}
    \toprule
    Masa (kg) & \multicolumn{3}{c}{Tiempo para 10 oscilaciones
    (s)} & Tiempo promedio (s) & Periodo $T$ (s) \\
    \cmidrule(lr){2-4}
    & Medición 1 & Medición 2 & Medición 3 &  &  \\
    \midrule
    0.05  & 16.52 & 16.5  & 16.78 & 16.60 & 1.660 \\
    0.10  & 16.53 & 16.68 & 17.09 & 16.76 & 1.677 \\
    0.15  & 16.68 & 16.70 & 17.00 & 16.79 & 1.679 \\
    0.20  & 16.68 & 16.71 & 16.95 & 16.78 & 1.678 \\
    0.25  & 16.91 & 16.77 & 16.74 & 16.81 & 1.681 \\
    \bottomrule
  \end{tabular}
\end{table}

\twocolumn

\subsection{Limitaciones}

% todo message in blue
\textcolor{workblue}{Se debe explicar el significado de los porcentajes de
error.}

Durante el experimento, se tomaron múltiples mediciones bajo las mismas
condiciones. Si el experimento hubiera estado libre de errores, estas
mediciones serían iguales. Sin embargo, los errores humanos e inexactitudes
inevitables resultan en una ligera variación. Algunas fuentes de
error posibles incluyen:

\begin{itemize}
  \item el efecto de rozamientos mínimos e imperceptibles sobre las
    oscilaciones del péndulo,
  \item longitudes mal medidas o inexactas,
  \item mediciones de tiempo inexactas\footnote{Algunos cronómetros
    empiezan a contar unas décimas de segundo después o antes.}
  \item y mediciones de ángulo inexactas o ángulos mal medidos.
\end{itemize}

Se elaboraron los cuadros~\ref{tab:error_longitud}~a~\ref{tab:error_amplitud}
para mostrar esta variación, que establece la existencia de errores e
inexactitudes.

\begin{table}[ht]
  \centering
  \begin{tabular}{cccc}
    \toprule
    Longitud $L$ (m) & \multicolumn{3}{c}{\% de error para $10$
    oscilaciones} \\
    \cmidrule(lr){2-4}
    & Med. 1 & Med. 2 & Med. 3  \\
    \midrule
    0.15 & 0.93\% & 0.31\% & 0.62\% \\
    0.30 & 2.63\% & 0.06\% & 2.28\% \\
    0.50 & 1.42\% & 0.71\% & 0.77\% \\
    1.00 & 0.19\% & 0.29\% & 0.43\% \\
    \bottomrule
  \end{tabular}
  \caption{Errores en la medición durante la variación de la
  longitud del hilo}\label{tab:error_longitud}
\end{table}

\begin{table}[ht]
  \centering
  \begin{tabular}{cccc}
    \toprule
    Masa $m$ (kg) & \multicolumn{3}{c}{\% de error para $10$
    oscilaciones} \\
    \cmidrule(lr){2-4}
    & Med. 1 & Med. 2 & Med. 3  \\
    \midrule
    0.05 & 0.48\% & 0.60\% & 1.08\% \\
    0.10 & 1.43\% & 0.54\% & 1.91\% \\
    0.15 & 0.68\% & 0.56\% & 1.23\% \\
    0.20 & 0.60\% & 0.42\% & 1.01\% \\
    0.25 & 0.59\% & 0.24\% & 0.42\% \\
    \bottomrule
  \end{tabular}
  \caption{Errores en la medición durante la variación de la
  masa de la pesa}\label{tab:error_masa}
\end{table}

\begin{table}[ht]
  \centering
  \begin{tabular}{cccc}
    \toprule
    Amplitud $\alpha$ ($^{\circ}$) & \multicolumn{3}{c}{\% de
      error para $10$
    oscilaciones} \\
    \cmidrule(lr){2-4}
    & Med. 1 & Med. 2 & Med. 3  \\
    \midrule
    6  & 1.09\% & 0.06\% & 1.15\% \\
    9  & 1.64\% & 1.60\% & 0.04\% \\
    12 & 1.79\% & 0.31\% & 1.73\% \\
    15 & 1.13\% & 0.24\% & 1.43\% \\
    18 & 0.85\% & 0.12\% & 1.03\% \\
    \bottomrule
  \end{tabular}
  \caption{Errores en la medición durante la variación de la
  amplitud de liberación}\label{tab:error_amplitud}
\end{table}

\section{Interpretación de los resultados}\label{sec:álgebra}

\subsection{Expectativas según la teoría}

El objeto de nuestro estudio es un péndulo simple, compuesto por una masa
suspendida al final de una cuerda. La cuerda se desplaza hasta una
cierta amplitud o ángulo y luego se libera. La velocidad del objeto
es un vector tangente a la trayectoria circular del péndulo y actúa
en el eje $x$, definido paralelo a $mg\sin{\alpha}$ como es visible
en la figura~\ref{fig:fuerzas}.

\begin{figure}[ht]
  \centering
  \begin{tikzpicture}
    \def\weight{2cm}
    \def\amplitude{35}

    % Tension

    \draw[->] (0,0) -- ({\weight * cot(90 + \amplitude)},2cm) node[near
    end,right]{$F_{T}$};

    % Weight and its components
    \draw[->] (0,0) -- (270:\weight) node[near end,right]{$mg$};

    \draw[->,dashed] (0,0) -- ({180 + \amplitude}:{\weight *
    sin(\amplitude)})
    node[pos=1,left,yshift=0mm]{$mg\sin{\alpha}$};

    \draw[->,dashed] (0,0) -- ({270 + \amplitude}:{\weight *
    cos(\amplitude)})
    node[pos=1,right,yshift=0mm]
    {$mg\cos{\alpha}$};

    % Mass
    \draw[fill=gray!30] (0,0) circle [radius=0.05cm];
  \end{tikzpicture}
  \caption{Diagrama de fuerzas del péndulo}\label{fig:fuerzas}
\end{figure}

\begin{figure}[ht]
  \centering
  \begin{tikzpicture}
    \def\weight{2cm}
    \def\amplitude{35}

    % Tension

    \draw[->] (0,0) -- ({\weight * cot(90 + \amplitude)},2cm)
    coordinate (A)
    node[near end, right, text=gray, font=\tiny]
    {$F_{T}$};

    % First showcase of angle \alpha
    \draw[dashed] (A) ++(0, -1.3mm) -- ++(0, -1cm);

    \def\angleradius{5mm}

    \draw (A) ++({cos(270) * \angleradius}, {sin(270) * \angleradius})
    arc[start angle=270, end angle={270 + \amplitude}, radius=\angleradius]
    node[pos=0.7, below]
    {$\alpha$};

    % Weight and its components
    \draw[->] (0,0) -- (270:\weight)
    node[pos=1,below,text=gray,font=\tiny]
    {$mg$};

    \draw[->,dashed] (0,0) -- ({180 + \amplitude}:{\weight *
    sin(\amplitude)})
    node[pos=0.5,left,yshift=1.4mm,text=gray,font=\tiny]
    {$mg\sin{\alpha}$};

    \draw[->,dashed] (0,0) -- ({270 + \amplitude}:{\weight *
    cos(\amplitude)})
    node[pos=0.5,right,yshift=1mm,text=gray,font=\tiny]
    {$mg\cos{\alpha}$};

    % Mass
    \draw[fill=gray!30] (0,0) circle [radius=0.05cm];

    % Second showcase of angle \alpha
    \draw (0,0) ++({cos(270 * \angleradius)}, {sin(270) * \angleradius})
    arc[start angle=270, end angle={270 + \amplitude}, radius=\angleradius]
    node[pos=0.7, below]
    {$\alpha$};
  \end{tikzpicture}
  \caption{Proyección del ángulo}
\end{figure}

La componente “tangencial” (u “horizontal” vista desde el marco de
referencia establecido) del peso es la fuerza que genera el
movimiento (velocidad tangencial) del péndulo, ya que es la única que
actúa a lo largo de la trayectoria en la que se mueve. Si bien existe
una componente vertical del peso, (denominada “componente radial”)
esta no influye en el movimiento del péndulo porque únicamente
mantiene la tensión en la cuerda, pero no influencia la velocidad
tangencial de la masa.

En un péndulo, la masa se encuentra en su punto más alto antes de ser
liberada, donde tiene su máxima energía potencial gravitatoria. A
medida que la masa comienza a descender, esa energía potencial se
convierte en energía cinética. Se asume que los efectos de la
resistencia del aire y otras fuerzas externas son despreciables.

En el punto más bajo de la trayectoria, toda la energía potencial se
ha transformado en cinética, lo que significa que la masa se mueve a
su mayor velocidad.

La tensión de la cuerda debe equilibrar o contrarrestar la componente
radial del peso para que la masa no caiga al suelo. Además, la
tensión es la fuerza centrípeta que genera la trayectoria circular de
la masa. Esta fuerza siempre va dirigida hacia el punto de
suspensión: el punto desde donde se encuentra colgada la cuerda.

La velocidad angular~\omega{} se define como el ángulo recorrido por
el objeto en un segundo, y se aproxima, cuando la amplitud~\alpha{} es
pequeña, mediante la siguiente formula en términos de $g$, la
aceleración gravitacional, y $l$, la longitud de la cuerda:

\begin{equation}
  \omega \approx \sqrt{\frac{g}{l}} \qquad\qquad \alpha < 15^{\circ}
\end{equation}

La frecuencia $f$ es la cantidad de oscilaciones que ocurren por
segundo, y se relaciona con la velocidad angular \omega{} mediante la
siguiente fórmula:

\begin{equation}
  \omega = 2 \pi f
\end{equation}

El periodo $T$ es el tiempo que tarda la masa en completar un ciclo,
definido como un movimiento de ida y vuelta. Este se calcula
mediante la siguiente ecuación:

\begin{equation}
  T \approx 2\pi\sqrt{\frac{l}{g}} \qquad\qquad \alpha \ll 1
  \label{eq:periodo}
\end{equation}

La demostración de la ecuación~\eqref{eq:periodo} se basa en una
breve manipulación algebraica:

\begin{align*}
  \omega &= 2\pi f \approx \sqrt{\frac{g}{l}} \\
  f &\approx \frac{1}{2\pi}\sqrt{\frac{g}{l}} \\
  T &= \frac{1}{f} \approx 2\pi\sqrt{\frac{l}{g}}
\end{align*}

Podemos concluir que el periodo es directamente proporcional a la raíz de la
longitud de la cuerda, o, equivalentemente, que el cuadrado del periodo es
proporcional a la longitud de la cuerda.

\begin{align}
  T &\propto \sqrt{l} \\
  T^{2} &\propto l
  \label{eq:periodo_proporcionalidad}
\end{align}

Según esta ecuación, el periodo del péndulo no depende de la masa del objeto que
cuelga: únicamente de la longitud de la cuerda y de la gravedad. Este hecho se
justifica a mayor detalle en la sección~\ref{sec:velocidad_tangencial}.

\subsection{Longitud y periodo}

\begin{figure}[ht]
  \centering
  \begin{tikzpicture}
    \begin{axis}[
        xlabel={Longitud $l$ (m)},
        ylabel={Periodo $T$ (s)},
        grid=both,
        xscale=0.85,
        legend style={at={({1/0.85},1.15)}},
        xmin=0, xmax=1.1,
        ymin=0, ymax=2.5,
        domain=0:1.1,
        samples=2,
      ]

      % Line of best fit
      \addplot[red] {1.2691*x + 0.8487};
      \addlegendentry{$y = 1.2691x + 0.8487$};

      % Data points
      \addplot[only marks, mark=o]
      coordinates {(0.15, 0.965) (0.30, 1.269) (0.50, 1.554)
      (1.00, 2.082)};
    \end{axis}
  \end{tikzpicture}
  \caption{Longitud $l$ (metros) contra periodo $T$
  (segundos)}\label{fig:longitud_periodo}
\end{figure}

\begin{figure}[ht]
  \centering
  \begin{tikzpicture}
    \begin{axis}[
        xlabel={Longitud $l$ (m)},
        ylabel={Cuadrado del periodo $T^{2}$ ($\text{s}^{2}$)},
        grid=both,
        xscale=0.85,
        legend style={at={({1/0.85},1.15)}},
        xmin=0, xmax=1.1,
        ymin=0, ymax=4.5,
        domain=0:1.1,
        samples=2,
      ]

      % Line of best fit
      \addplot[red] {3.9696*x + 0.3873};
      \addlegendentry{$y = 3.9696x + 0.3873$};

      % Data points
      \addplot[only marks, mark=o]
      coordinates {(0.15, 0.93123) (0.3, 1.60952) (0.5, 2.41595)
      (1.00, 4.33334)};
    \end{axis}
  \end{tikzpicture}
  \caption{Longitud $l$ (metros) contra el cuadrado del periodo $T^{2}$
  (segundos cuadrados)}\label{fig:longitud_cuadrado_periodo}
\end{figure}

La gráfica~\ref{fig:longitud_periodo} es consistente con una
correlación positiva entre la longitud de la cuerda y el periodo del
péndulo: a medida que la longitud aumenta, el periodo también lo hace. Sin
embargo, los puntos no se ajustan exactamente a una relación lineal.

Según la teoría,\footnote{Vea la
  ecuación~\eqref{eq:periodo_proporcionalidad} en la
página~\pageref{eq:periodo_proporcionalidad}.} esto ocurre porque el periodo es
proporcional a la raíz cuadrada de la longitud, no a la longitud
directamente: la relación no es lineal. Por eso, en la
gráfica~\ref{fig:longitud_cuadrado_periodo}, se representó el
cuadrado del periodo frente a la longitud. En este caso, los puntos
se ajustaron a una línea recta, demostrando la relación lineal entre
el cuadrado del periodo y la longitud de la cuerda.

Se cree que lo anterior se debe a que, al aumentar la longitud de la cuerda, la
trayectoria que el objeto recorre también aumenta. El objeto toma más
tiempo en recorrer una distancia mayor, lo que provoca que el periodo
(el tiempo para una oscilación completa) se alargue.

\subsubsection{Modelado matemático}

La relación que se observa en la gráfica~\ref{fig:longitud_cuadrado_periodo}
puede ser modelada mediante una regresión lineal. La ecuación resultante está
dada dentro de la gráfica, y es la siguiente:

\begin{equation}
  T^{2} = 3.9696l + 0.3873
\end{equation}

Esto permite predecir el periodo de un péndulo simple a partir de la longitud de
su cuerda, de una forma consistente con la teoría.

La ecuación teórica~\eqref{eq:periodo} puede manipularse para obtener una forma
similar a la ecuación de regresión:

\begin{align*}
  T &\approx 2\pi\sqrt{\frac{l}{g}} \\
  T^{2} &\approx 4\pi^{2}\frac{l}{g} \\
  T^{2} &\approx \frac{4\pi^{2}}{9.81\,\text{m/s$^{2}$}}l \\
  T^{2} &\approx 4.03\,\text{s$^{2}$/m} \cdot l
\end{align*}

El coeficiente 4.03 de la ecuación teórica es muy cercano al 3.97 obtenido
mediante la regresión lineal de los datos experimentales. Esta similitud
entre la teoría y los resultados experimentales fortalece la validez de
nuestras conclusiones.

La pequeña diferencia entre ambos valores puede atribuirse a varios factores
experimentales, como la resistencia del aire, pequeñas imperfecciones en la
medición del tiempo y la longitud, o ligeras desviaciones de la verticalidad en
el montaje del péndulo. Lo mismo aplica para la presencia de un término
independiente (0.3873) en la ecuación experimental, que también sugiere la
existencia de errores sistemáticos en las mediciones.

\subsection{Amplitud y periodo}

\begin{figure}[ht]
  \centering
  \begin{tikzpicture}
    \begin{axis}[
        xlabel={Amplitud $\alpha$ ($^{\circ}$)},
        ylabel={Periodo $T$ (s)},
        grid=both,
        scale=0.5,
        legend style={at={(1,1.3)}},
        grid=major,
        xmin=0, xmax=20,
        ymin=0, ymax=3,
        domain=0:20,
        samples=2,
      ]

      % Line of best fit
      \addplot[red] {0.0002*x + 1.6655};
      \addlegendentry{$y = 0.0002x + 1.6655$};

      % Data points
      \addplot[only marks, mark=o]
      coordinates {(6, 1.657) (9, 1.668) (12, 1.680)
      (15, 1.676) (18, 1.655)};
    \end{axis}
  \end{tikzpicture}
  \caption{Amplitud $\alpha$ (grados) contra periodo $T$
  (segundos)}\label{fig:amplitud_periodo}
\end{figure}

La gráfica~\ref{fig:amplitud_periodo} muestra la relación entre la
amplitud y el periodo. La variación entre los puntos es mínima. Al
añadir una línea de tendencia, la ecuación resultante muestra una
pendiente de $0.0002$\,s, lo que indica una variación insignificante
en los datos. De manera similar, la gráfica refleja que el cambio en
el periodo al aumentar la amplitud es muy pequeño.

La ausencia de una relación clara es consistente con la
ecuación~\eqref{eq:periodo}, la cual asocia el periodo únicamente con la
aceleración gravitacional y la longitud de la cuerda.

Los resultados experimentales sugieren que el periodo de un péndulo simple es
independiente de la amplitud de la oscilación. En la regresión lineal, se cree
que el término independiente es consecuencia de los factores que se mantuvieron
constantes durante el experimento, como la longitud de la cuerda y la
aceleración gravitacional.

\subsubsection{Energía potencial}

Esta sección ofrece una posible explicación para los resultados
observados, basándose en el análisis energético.

La diferencia de la energía potencial del objeto entre pequeñas amplitudes
es mínima, lo que puede ayudar a explicar que la amplitud no afecte de manera
significativa el tiempo que el objeto tarda en completar una oscilación.

Se asume que los efectos de la resistencia del aire y otras fuerzas
externas son despreciables, por lo que se realiza el análisis
basándose en la suposición de que toda la energía potencial gravitatoria
se convierte en energía cinética.

Debido a que una mayor energía cinética implica una mayor velocidad
máxima en la oscilación, podría esperarse un menor periodo
oscilatorio, incluso a pesar de la mayor longitud del arco a través
del que el movimiento ocurre cuando se aumenta la amplitud. Sin
embargo, las diferencias en energía potencial gravitatoria durante
el experimento son mínimas.

Se puede calcular la altura inicial $h$ de una pesa que se ata a un hilo de
longitud $l$ y se libera con la amplitud $\alpha$ según la siguiente fórmula:

\begin{align}
  h &= l(\sin{(270^{\circ} + \alpha)} - \sin{270^{\circ}}) \nonumber \\
  &= l(\sin{(270^{\circ} + \alpha)} + 1) \nonumber \\
  &= l(-\cos{\alpha} + 1) \nonumber \\
  &= l(1 - \cos{\alpha})
\end{align}

También se puede calcular la energía potencial gravitatoria de la pesa
a partir de su masa $m$ y su altura inicial $h$:

\begin{equation}
  U = mgh
\end{equation}

Combinando ambas fórmulas, se puede estimar la energía potencial
gravitatoria de la pesa al liberarse a las distintas amplitudes, como
se ve en la figura~\ref{fig:energía_amplitud}.

\begin{figure}[ht]
  \centering
  \begin{tikzpicture}
    \begin{axis}[
        scale=0.5,
        xlabel={Amplitud $\alpha$ ($^{\circ}$)},
        ylabel={Energía potencial $T$ (J)},
        grid=major,
        xmin=0, xmax=20,
        ymin=0, ymax=0.1,
        xtick={6, 9, 12, 15, 18},
        domain=0:20,
        samples=200,
      ]
      \addplot[mark=none, color=blue]{0.5 * 9.81 * 0.3 * (1 - cos(x))};
    \end{axis}
  \end{tikzpicture}
  \caption{Energía potencial (julios) de una masa de $0.5$\,kg
    liberada a un radio de $0.3$\,m contra amplitud de liberación $\alpha$
  (grados)}\label{fig:energía_amplitud}
\end{figure}

Los cálculos muestran que la diferencia entre las energías cinéticas
para variaciones en pequeños valores de amplitud es pequeña.

\subsection{Masa y periodo}

\begin{figure}[ht]
  \centering
  \begin{tikzpicture}
    \begin{axis}[
        scale=0.5,
        xlabel={Masa $m$ (kg)},
        ylabel={Periodo $T$ (s)},
        grid=major,
        legend style={at={(1,1.3)}},
        xmin=0, xmax=0.3,
        ymin=0, ymax=2,
        domain=0:0.3,
        samples=2,
      ]

      % Line of best fit
      \addplot[red] {0.0043*x + 1.6621};
      \addlegendentry{$y = 0.0043x + 1.6621$};

      % Data points
      \addplot[only marks, mark=o]
      coordinates {
        (0.05, 1.660)
        (0.10, 1.677)
        (0.15, 1.679)
        (0.20, 1.678)
        (0.25, 1.681)
      };
    \end{axis}
  \end{tikzpicture}
  \caption{Masa $m$ (kilogramos) contra periodo $T$
  (segundos)}\label{fig:masa_periodo}
\end{figure}

La gráfica~\ref{fig:masa_periodo} muestra la relación entre la masa y
el periodo. En la gráfica, se observa que la variación entre los
puntos es mínima. La línea de tendencia tiene la ecuación $T =
0.0043m + 1.6621$,
donde el coeficiente $0.0043$\,s/kg representa el cambio en el periodo por cada
kilogramo adicional de masa. Este valor es extremadamente pequeño, lo que indica
que la masa tiene un efecto insignificante en el periodo.

El término independiente $1.6621$\,s representa el periodo base del péndulo,
determinado principalmente por la longitud de la cuerda que se mantuvo constante
durante esta parte del experimento. Las pequeñas desviaciones de los puntos
respecto a la línea de tendencia podrían atribuirse a errores experimentales.

Según el análisis teórico, y como es discutido en
la sección~\ref{sec:velocidad_tangencial}, el periodo es independiente de la
masa. Esto también es consistente con la fórmula~\eqref{eq:periodo}.

\subsection{Discusión de las preguntas orientadoras}

\subsubsection*{¿Qué sucede con la velocidad tangencial cuando
aumenta la fuerza centrípeta?}\label{sec:velocidad_tangencial}

La fuerza centrípeta en un movimiento circular se define por la
siguiente formula:

\begin{equation}
  F_{c} = \frac{mv^{2}}{r}
\end{equation}

Podemos concluir que la fuerza centrípeta $F_{c}$ y el cuadrado de la
velocidad del objeto en movimiento $v^{2}$ son proporcionales. Una
mayor fuerza centrípeta, entonces, implica una mayor velocidad,
siempre y cuando otros factores como la masa y la longitud (radio) se
mantengan iguales.

En el péndulo, la fuerza centrípeta asociada al movimiento circular
proviene de la tensión de la cuerda. Sin embargo, la tensión en sí
misma varía a lo largo de la trayectoria. En el punto más bajo del
recorrido, la tensión es mayor porque no solo proporciona la fuerza
necesaria para producir el movimiento circular, sino que también debe
compensar totalmente el peso del péndulo, que en ese momento actúa en
la dirección opuesta. Dado que la tensión es variable, la velocidad
del péndulo también cambia. En el punto más bajo, tanto la tensión
como la velocidad alcanzan sus máximos.

\subsubsection*{¿Cómo se puede aumentar la velocidad tangencial?}

La velocidad máxima del péndulo ocurre cuando está en su punto más
bajo y su energía gravitacional potencial se ha convertido totalmente
en energía cinética.

La altura del péndulo relativa a su punto más bajo es descrita por la fórmula:

\begin{align}
  h &= l(\cos(0) - \cos(\alpha)) \nonumber \\
  &= l(1 - \cos(\alpha))
\end{align}

Donde $l$ se refiere a la longitud de la cuerda.

Por eso, sabemos que:

\begin{equation}
  h \propto l
\end{equation}

Por otro lado, la velocidad del péndulo en su punto más bajo es dada por:

\begin{align}
  \frac{1}{2}m v^{2} &= mgh \\
  v^{2} &= 2gh
\end{align}

Sustituyendo $h$,

\begin{equation}
  v^{2} = 2gl(1 - \cos(\alpha))
\end{equation}

Podemos concluir que:

\begin{equation}
  v^{2} \propto l
\end{equation}

También,

\begin{equation}
  v^{2} \propto 1 - \cos(\alpha)
\end{equation}

En otras palabras, aumentar el ángulo ($0 \leq \alpha \leq
180^{\circ}$) o aumentar la longitud de la cuerda resulta en una
mayor velocidad tangencial.

La masa no tiene ningún efecto. Durante la manipulación algebraica,
la masa $m$ se presenta como un factor en ambos lados de la ecuación,
así que se obtienen fórmulas en términos de otras variables independientes.

\subsubsection*{¿Qué sucede con el periodo del péndulo si la cuerda
encuentra un obstáculo?}

En el sistema, la pesa se desplaza a lo largo de un arco cuyo origen
se llama \emph{pivote}. Cuando no hay un obstáculo que interrumpa la
oscilación, el pivote no cambia y es el punto desde el que está
fijada la cuerda.

Sin embargo, cuando la cuerda se encuentra con un obstáculo, el
pivote se convierte en el punto donde ocurre el contacto. Eso
significa que por una parte de la oscilación, la pesa sigue la
trayectoria de un arco cuyo radio es menor. Una vez que el péndulo
supera el obstáculo, se mueve otra vez desde su pivote original.

\begin{figure}[h]
  \centering
  \begin{tikzpicture}
    \draw (0,0) -- (-55:3.5cm) coordinate (A) -- ++(-115:1.3cm)
    coordinate (C);
    \node[left, xshift=-2mm] at (A) {nuevo pivote};

    \path (A) ++(-0.67mm, 0) coordinate (B);
    \fill (B) circle (0.75mm);

    \def\masswidth{5mm};
    \def\massheight{2mm};

    \draw
    (C) --
    ++({-115 - 90}:{\masswidth/2}) --
    ++({-115}:{\massheight}) coordinate (D) --
    ++({-115 + 90}:\masswidth) --
    ++({-115 + 90 * 2}:\massheight) --
    ++({-115 - 90}:{\masswidth / 2}) node[right, xshift=3mm]{masa};

    \draw
    (D) --
    ++({-115 - 90}:1.5mm) --
    ++({-115 + 90}:{\masswidth + 1.5mm * 2});
  \end{tikzpicture}
  \caption{El sistema cuando la cuerda se encuentra con un objeto}
\end{figure}

Se asume que los efectos de la resistencia del aire y otras fuerzas
externas son despreciables, por lo que no existe fuerza que disminuya
la velocidad tangencial de la pesa además de la gravedad. Sin
embargo, el recorrido total disminuye porque una parte del arco
original es reemplazada por un arco más pequeño. De por sí, este
hecho ofrece un fundamento fuerte para creer que un obstáculo
disminuiría el periodo.

Por otro lado, la fórmula~\eqref{eq:periodo} de la
página~\pageref{eq:periodo} indica que el periodo de un péndulo
depende de la longitud de la cuerda y la gravedad. En partiular, el
periodo y la longitud de la cuerda tienen una correlación positiva.

La fórmula es válida para un cuerpo que se libera sin una velocidad
inicial, pero es útil para nuestro análisis porque se cree que la velocidad
otorgada a la pesa por su caída parcial hasta este punto tiene un
efecto mínimo comparable al de la amplitud \alpha.

Podemos afirmar con suficiente confianza que al reducirse la longitud
efectiva de la cuerda, el periodo de la oscilación pequeña será mucho
menor en comparación con el periodo que habría de no haber ningún obstáculo.

En otras palabras, durante el tiempo en el que la pesa oscila con
una longitud de cuerda menor, recorrería una parte de la trayectoria
más rápido. Al final, todo el movimiento es más rápido de lo que
sería si no hubiera encontrado el obstáculo.

Por todo lo anterior, se concluye que el periodo disminuiría si la
cuerda se encontrara con un obstáculo.

\subsection*{Suponga que el péndulo roza un líquido en cada
oscilación. ¿Cómo se vería la gráfica del espacio contra el tiempo?}

\begin{figure}[ht]
  \centering
  \begin{tikzpicture}
    \begin{axis}[
        xlabel={Tiempo},
        ylabel={Posición horizontal},
        grid=both,
        xmin=0, xmax={360*4},
        ymin=-1, ymax=1,
        domain=0:{360*4},
        xtick=\empty,
        ytick={0},
        samples=300,
      ]
      \addplot[red] {cos(x) * (360 * 4 - x)/(360 * 4)};
    \end{axis}
  \end{tikzpicture}
  \caption{El comportamiento general esperado del
  movimiento}\label{fig:líquido}
\end{figure}

\paragraph{Descenso inicial:} Al inicio, el péndulo
parte de una altura determinada. Mientras baja hacia el punto más
bajo de su trayectoria, donde $x=0$, su velocidad aumenta, lo que se
refleja en una línea cada vez más empinada (una mayor pendiente)
en la gráfica. La pendiente de la gráfica posición-tiempo equivale a la
velocidad del péndulo.

\paragraph{En el punto más bajo $x=0$ y al entrar en el líquido:} Al llegar al
punto más bajo y entrar en contacto con el líquido, la fricción
comienza a actuar, reduciendo la velocidad. Esto genera una
disminución de la pendiente en la gráfica, ya que la velocidad se
reduce debido a la resistencia del líquido.

\paragraph{Ascenso posterior:} Al subir después de pasar por el líquido, el
péndulo pierde velocidad, ya que su energía cinética se
convierte en energía potencial y, además, cabe mencionar que la
fricción del líquido ha reducido parte de su energía. La pendiente
sigue disminuyendo hasta que, en el punto más alto de su trayectoria,
la velocidad es cero, por lo que la pendiente también será cero. Sin
embargo, debido a la pérdida de energía, la altura alcanzada en este
punto es menor que la altura inicial, lo que se refleja en la gráfica
con un valor menor que el de partida.

\paragraph{Nuevo descenso hacia el punto más bajo nuevamente:} Al
descender desde el nuevo punto más alto, que es menor al inicial, la
velocidad del péndulo aumenta nuevamente. Esto se refleja en la
gráfica como una pendiente creciente, ya que al moverse hacia el
punto más bajo, la velocidad aumenta. Sin embargo, al pasar por el
líquido nuevamente, la fricción hace que la velocidad disminuya, lo
cual disminuye la pendiente de la gráfica. Al subir hacia el nuevo
punto más alto, la energía cinética se vuelve a convertir en energía
potencial, y la velocidad disminuye.

Debido a la fricción, el péndulo pierde energía en cada oscilación,
por lo que la altura que alcanza es cada vez menor. Esto se refleja
en la gráfica con una amplitud decreciente, ya que el péndulo no
regresa al punto de partida original. Cada nuevo punto mínimo o
máximo en la gráfica de posición-tiempo es más bajo que el anterior.

\subsection{Comparación con el movimiento armónico simple}

El movimiento armónico simple es una oscilación en la que la fuerza restauradora
es proporcional al desplazamiento. En el caso del péndulo, la fuerza
restauradora es dada por:

\begin{equation}\label{eq:restauradora}
  F_{r} = -mg\sin{\alpha}
\end{equation}

Donde $m$ es la masa del objeto, $g$ es la aceleración gravitacional y $\alpha$
es el ángulo de desplazamiento.

Aunque la ecuación~\eqref{eq:restauradora} no es lineal, es muy cercana
a una para pequeñas amplitudes.

\begin{equation}
  \sin{\alpha} \approx \alpha \qquad\qquad \alpha < 15^{\circ}
\end{equation}

Esto permite obtener una fórmula lineal que aproxima la fuerza restauradora:

\begin{equation}
  F_{r} \approx -mg\alpha \qquad\qquad \alpha < 15^{\circ}
\end{equation}

Debido a que la fuerza restauradora es proporcional al desplazamiento, el
péndulo se comporta como un movimiento armónico simple para pequeñas amplitudes.

Esta idea es consistente con los resultados obtenidos en el experimento.

\begin{figure}[ht]
  \begin{tikzpicture}
    \begin{axis}[
        xlabel={Tiempo $t$ (s)},
        ylabel={Posición horizontal $x$ (m)},
        legend style={at={({1/0.85},1.20)}},
        grid=both,
        xmin=0, xmax={3},
        xscale=0.85,
        ymin=-0.3, ymax=0.3,
        domain=0:{16.6},
        samples=1000,
      ]
      \addplot[red] {0.039 * cos((360 * x)/0.965)};
      \addlegendentry{Longitud 0.15 m}

      \addplot[blue] {0.259 * cos((360 * x)/2.08)};
      \addlegendentry{Longitud 1.00 m}
    \end{axis}
  \end{tikzpicture}
  \caption{El desplazamiento en función del tiempo para un ángulo
  inicial de $15^{\circ}$}\label{fig:sinusoide}
\end{figure}

\begin{figure}[ht]
  \centering
  \begin{tikzpicture}
    \begin{axis}[
        xlabel={Tiempo $t$ (s)},
        ylabel={Velocidad $x$ (m/s)},
        legend style={at={({1/0.85},1.20)}},
        grid=both,
        xmin=0, xmax={4},
        xscale=0.85,
        ymin=-1, ymax=1,
        domain=0:{16.6},
        samples=1000,
      ]
      \addplot[red] {-0.253 * sin((360 * x)/0.965)};
      \addlegendentry{Longitud 0.15 m}

      \addplot[blue] {-0.781 * sin((360 * x)/2.08)};
      \addlegendentry{Longitud 1.00 m}
    \end{axis}
  \end{tikzpicture}
  \caption{La velocidad en función del tiempo para el ángulo de
  $15^{\circ}$}\label{fig:velocidad_MAS}
\end{figure}

\begin{figure}[ht]
  \centering
  \begin{tikzpicture}
    \begin{axis}[
        xlabel={Tiempo $t$ (s)},
        ylabel={Aceleración $a$ (m/s$^{2}$)},
        legend style={at={({1/0.85},1.20)}},
        grid=both,
        xmin=0, xmax={4},
        xscale=0.85,
        ymin=-2.5, ymax=2.5,
        domain=0:{16.6},
        samples=1000,
      ]
      \addplot[red] {-1.645 * cos((360 * x)/0.965)};
      \addlegendentry{Longitud 0.15 m}

      \addplot[blue] {-2.358 * cos((360 * x)/2.08)};
      \addlegendentry{Longitud 1.00 m}
    \end{axis}
  \end{tikzpicture}
  \caption{La aceleración en función del tiempo para la longitud de
  0.15~m y el ángulo de $15^{\circ}$}\label{fig:aceleración_MAS}
\end{figure}

\begin{figure}[ht]
  \centering
  \begin{tikzpicture}
    \begin{axis}[
        xlabel={Tiempo $t$ (s)},
        ylabel={Posición horizontal $x$ (m)},
        legend style={at={({1/0.85},1.23)}},
        grid=both,
        xmin=0, xmax={3},
        xscale=0.85,
        ymin=-0.3, ymax=0.3,
        domain=0:{16.6},
        samples=1000,
      ]
      \addplot[red] {0.063 * cos(360 * x / 1.657)};
      \addlegendentry{Ángulo de $6^{\circ}$}

      \addplot[blue] {0.155 * cos(360 * x / 1.6763)};
      \addlegendentry{Ángulo de $15^{\circ}$}
    \end{axis}
  \end{tikzpicture}
  \caption{El desplazamiento en función del tiempo para una longitud
  de cuerda de 0.6\,m\label{fig:desplazamiento_amplitud}}
\end{figure}

\begin{figure}[ht]
  \centering
  \begin{tikzpicture}
    \begin{axis}[
        xlabel={Tiempo $t$ (s)},
        ylabel={Velocidad $x$ (m/s)},
        legend style={at={({1/0.80},1.25)}},
        grid=both,
        xmin=0, xmax={4},
        xscale=0.80,
        ymin=-1, ymax=1,
        domain=0:{16.6},
        samples=1000,
      ]
      \addplot[red] {-0.238 * sin(360 * x / 1.657)};
      \addlegendentry{Ángulo de $6^{\circ}$}

      \addplot[blue] {-0.582 * sin(360 * x / 1.6763)};
      \addlegendentry{Ángulo de $15^{\circ}$}
    \end{axis}
  \end{tikzpicture}
  \caption{La velocidad en función del tiempo para una longitud de
  0.6\,m\label{fig:velocidad_amplitud}}
\end{figure}

\begin{figure}[ht]
  \centering
  \begin{tikzpicture}
    \begin{axis}[
        xlabel={Tiempo $t$ (s)},
        ylabel={Aceleración $a$ (m/s$^{2}$)},
        legend style={at={({1/0.85},1.25)}},
        grid=both,
        xmin=0, xmax={4},
        xscale=0.85,
        ymin=-2.5, ymax=2.5,
        domain=0:{16.6},
        samples=1000,
      ]
      \addplot[red] {-0.901 * cos(360 * x / 1.657)};
      \addlegendentry{Ángulo de $6^{\circ}$}

      \addplot[blue] {-2.180 * cos(360 * x / 1.6763)};
      \addlegendentry{Ángulo de $15^{\circ}$}
    \end{axis}
  \end{tikzpicture}
  \caption{La aceleración en función del tiempo para la longitud de
  0.15\,m\label{fig:aceleración_amplitud}}
\end{figure}

% TODO: Finish this section

El periodo esperado para un movimiento armónico simple es muy cercano a los
periodos obtenidos en el experimento. Al mismo tiempo, el análisis energético da
velocidades máximas consistentes con las esperadas de un movimiento armónico
simple.

La figura~\ref{fig:periodo_mas} compara los periodos obtenidos en el experimento
con los periodos esperados para un movimiento armónico simple. La variación
entre los valores esperados y los reales es mínima, lo que sugiere
que el péndulo se comporta como un movimiento armónico simple.

% TODO: Create this figure
\begin{figure}[ht]
  \centering
  \begin{tikzpicture}
  \end{tikzpicture}
  \caption{Periodo $T$ medido y esperado para un movimiento armónico
  simple en función de la longitud de la cuerda $l$}\label{fig:periodo_mas}
\end{figure}

% TODO: Create a table with calculated vs. measured values, together
% with percentage variation

\begin{table}
  \caption{Comparación entre los valores medidos y esperados para $T$
  según el movimiento armónico simple, con porcentajes de variación}
\end{table}

La figura~\ref{fig:sinusoide} muestra el desplazamiento en función del tiempo
de un movimiento armónico con las mismas características que una de las
configuraciones experimentales.

\subsection{Conclusiones}

El experimento permitió evaluar la validez de las hipótesis.
En primer lugar, se planteó que existía una correlación positiva
entre la longitud de la cuerda y la duración de las oscilaciones, lo
cual fue confirmado. Esto se evidenció tanto en la gráfica de
resultados, que mostró una pendiente que reflejaba dicha relación,
como en la tendencia observada en los datos obtenidos. En segundo
lugar, se planteó la hipótesis de que la variación de la masa es
irrelevante para el período de oscilación del péndulo. Esta hipótesis
fue comprobada mediante la gráfica de resultados, que mostró una
pendiente muy mínima y cercana a 0,\footnote{Posiblemente debida a errores
experimentales.} lo que sugiere que no existe una relación
significativa entre las dos variables. En tercer lugar, se hipotetizó
que existe una correlación positiva entre la amplitud y la velocidad
máxima del péndulo durante una oscilación. Sin embargo, dado a las
pequeñas amplitudes que se tomaron para el experimento, y los
resultados muy cercanos los unos a los otros, no se pudo establecer
experimentalmente una relación directa entre ambas variables.

\renewcommand{\refname}{Bibliografía}
\begin{thebibliography}{1}
  \bibitem{}
  Giancoli, D.\,C. (2014) \textit{Physics: Principles and
  Applications}. Pearson Education, Inc.
\end{thebibliography}

\end{document}
